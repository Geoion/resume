% This work may be distributed and/or modified under the
% conditions of the LaTeX Project Public License version 1.3c,
% available at http://www.latex-project.org/lppl/.


\documentclass[11pt,a4paper,article]{moderncv}   % possible options include font size ('10pt', '11pt' and '12pt'), paper size ('a4paper', 'letterpaper', 'a5paper', 'legalpaper', 'executivepaper' and 'landscape') and font family ('sans' and 'roman')

% moderncv 主题
\moderncvstyle{casual}                        % 选项参数是 ‘casual’, ‘classic’, ‘oldstyle’ 和 ’banking’
\moderncvcolor{black}                         % 选项参数是 blue(默认)、orange、green、red、purple、grey、black
%\nopagenumbers{}                             % 消除注释以取消自动页码生成功能

% 字符编码
\setmainfont{SimSun} 

% 调整页面边距
\usepackage[scale=0.8]{geometry}
\setlength{\hintscolumnwidth}{3cm}

% 个人信息
\name{陈}{成}
%\title{简历题目}                     		% 可选项、如不需要可删除本行
%\address{街道及门牌号}{邮编及城市}         % 可选项、如不需要可删除本行
\phone[mobile]{+86~185~2022~2024}           % 可选项、如不需要可删除本行
%\phone[fixed]{+2~(345)~678~901}            % 可选项、如不需要可删除本行
%\phone[fax]{+3~(456)~789~012}              % 可选项、如不需要可删除本行
\email{chenfjm@gmail.com}                   % 可选项、如不需要可删除本行
\homepage{chencheng.sn.cn}                  % 可选项、如不需要可删除本行
\social[github]{chenfjm}
\social[linkedin]{chenfjm}
%\extrainfo{附加信息 (可选项)}                 % 可选项、如不需要可删除本行
%\photo[64pt][0.4pt]{picture}                  % ‘64pt’是图片必须压缩至的高度、‘0.4pt‘是图片边框的宽度 (如不需要可调节至0pt)、’picture‘ 是图片文件的名字;可选项、如不需要可删除本行

%----------------------------------------------------------------------------------
%            内容
%----------------------------------------------------------------------------------
\begin{document}
\makecvfooter								  % cvfooter,cvtitle,title,footer

\section{基本信息}
\cvdoubleitem{姓~~~~名}{\small 陈成}{家庭地址}{\small 陕西省延安市}
\cvdoubleitem{性~~~~别}{\small 男}{出生日期}{\small 1990-03-03}
\cvdoubleitem{学~~~~校}{\small 西安理工大学}{专~~~~业}{\small 计算机科学与技术}
\cvdoubleitem{学~~~~历}{\small 学士}{毕业时间}{\small 2013-07}
\cvdoubleitem{求职意向}{\small 软件开发}{工作地点}{\small 北京}

\section{专业技能}
\cvitem{编程语言}{\small C,C++,Python}
\cvitem{开发工具}{\small git,svn,gvim,cmake,gcc/g++,gdb,pdb,doxygen,visio}
\cvitem{第三方库}{\small Boost,libevent,Qt,Duilib,Django}
\cvitem{}{\small 分析过Linux内核源码,深刻理解操作系统基本原理,熟练使用嵌入式Linux开发
	相关工具。熟悉面向对象,网络编程,并发编程。}

\section{工作经验}
\cvdoubleitem{时~~~~间}{\small 2013.02-2014.07}{公司名称}{\small 广州致远电子有限公司}
\cvitem{职位名称}{\small 软件研发工程师}

\section{项目经验}
\vspace{3pt}
\subsection{方圆电器能效测量系统}
\cvitem{项目时间}{\small 2014.03-2014.07}
\cvitem{项目描述}{\small 通过实时读取测量设备的数据来完成电器功耗的计算、波形的显示、报表导出。}
\cvitem{责任描述}{\small 根据客户提出的需求独立完成软件的设计与实现。}

\vspace{3pt}
\subsection{功率分析仪客户端管理软件}
\cvitem{项目时间}{\small 2013.07-2014.02}
\cvitem{项目描述}{\small 功率分析仪的辅助软件,帮助客户在PC端操作功率分析仪,并对测试数据进行分
		析、显示、导出报表等操作。}
\cvitem{责任描述}{\small 网络通信模块、报表模块。}

\vspace{3pt}
\subsection{基于wince平台的DirectUI界面库实现}
\cvitem{项目时间}{2013.02-2013.05}
\cvitem{项目描述}{\small 根据xml描述实现界面的绘制,将用户界面与业务逻辑分离。}
\cvitem{责任描述}{\small 独立完成项目的需求、设计、编码实现、测试。}

\vspace{3pt}
\subsection{停车位搜索系统}
\cvitem{项目时间}{\small 2012.07-2012.09}
\cvitem{项目描述}{\small 通过zigbee和GPRS无线网络将停车场的空闲车位信息上传到服务器,用户通过手
		机与服务器通信来查找周围空闲的停车位。}
\cvitem{责任描述}{\small 项目的整体规划与设计,后端系统的实现。}
\cvitem{项目成果}{\small 实用新型专利一项,软件著作权一项。}
%\renewcommand{\listitemsymbol}{-}             % 改变列表符号

% 来自BibTeX文件但不使用multibib包的出版物
%\renewcommand*{\bibliographyitemlabel}{\@biblabel{\arabic{enumiv}}}% BibTeX的数字标签
\nocite{*}
\bibliographystyle{plain}
\bibliography{publications}                    % 'publications' 是BibTeX文件的文件名

% 来自BibTeX文件并使用multibib包的出版物
%\section{出版物}
%\nocitebook{book1,book2}
%\bibliographystylebook{plain}
%\bibliographybook{publications}               % 'publications' 是BibTeX文件的文件名
%\nocitemisc{misc1,misc2,misc3}
%\bibliographystylemisc{plain}
%\bibliographymisc{publications}               % 'publications' 是BibTeX文件的文件名

\end{document}

